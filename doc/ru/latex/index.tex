\subsubsection*{Класс \hyperlink{classmpiworker_1_1MPIWorker}{mpiworker\-::\-M\-P\-I\-Worker}}

Содержит методы-\/обертки с простым синтаксисом для коллективных операций \href{https://www.open-mpi.org/}{\tt M\-P\-I} (Scatterv, Gatherv и других). Данные храняться в векторах \href{http://ru.cppreference.com/w/cpp/container/vector}{\tt std\-::vector}, работа выполняется в коммуникаторе M\-P\-I\-::\-C\-O\-M\-M\-\_\-\-W\-O\-R\-L\-D.

Для выполнения коллективных операций требуется создать объект класса M\-P\-I\-Worker\-: 
\begin{DoxyCode}
\hyperlink{classmpiworker_1_1MPIWorker}{mpiworker::MPIWorker} w;
\end{DoxyCode}
 затем установить режим работы и общее число обрабатываемых элементов 
\begin{DoxyCode}
w.\hyperlink{classmpiworker_1_1MPIWorker_a2ff88f266efae23ec2a12a56bd0472d1}{setMode}(0);
w.\hyperlink{classmpiworker_1_1MPIWorker_afcce321227c6d15a5fc2145cf59cd54d}{setNElems}(9999);
\end{DoxyCode}


Пример разделения элементов вектора на приблизительно равные части\-: 
\begin{DoxyCode}
\textcolor{comment}{// result for nNodes = 3}
                                                   \textcolor{comment}{// that is mpirun -np 3 ./example\_scatter\_then\_gather}

  \hyperlink{classmpiworker_1_1MPIWorker}{mpiworker::MPIWorker} w;                          \textcolor{comment}{// MPI::Init(), Get\_size() and
       Get\_rank()}

  \textcolor{keywordtype}{int} N = 0; 
  std::vector<float> x; 

  \textcolor{keywordflow}{if}( !w.\hyperlink{classmpiworker_1_1MPIWorker_acbd3bd07d15ffa90a2c112edeecee6e0}{getRankNode}() ) 
  \{ 
      N = 11; 
      x.resize(N);
      std::iota(x.begin(),x.end(),1);              \textcolor{comment}{// rank=0: x = \{1, 2, 3, 4, 5, 6, 7, 8, 9, 10, 11\}}
  \}

  w.\hyperlink{classmpiworker_1_1MPIWorker_ae22bafa8bd4d6515e5b91ef95518c87d}{bcast}<\textcolor{keywordtype}{int}>(N,MPI::INT);                        \textcolor{comment}{// N == 11 on all ranks}
  w.\hyperlink{classmpiworker_1_1MPIWorker_a2ff88f266efae23ec2a12a56bd0472d1}{setMode}(1);                                    \textcolor{comment}{// all nodes have equal rights}
  w.\hyperlink{classmpiworker_1_1MPIWorker_afcce321227c6d15a5fc2145cf59cd54d}{setNElems}(N);                                  \textcolor{comment}{// counts = \{ 3, 4, 4 \}}
                                                   \textcolor{comment}{// displacement  = \{ 0, 3, 7 \}}
  \textcolor{keywordflow}{if}( !w.\hyperlink{classmpiworker_1_1MPIWorker_acbd3bd07d15ffa90a2c112edeecee6e0}{getRankNode}() ) w.\hyperlink{classmpiworker_1_1MPIWorker_ab9f20357773fe10fbe3bc6d92754d4e0}{print}();

  std::vector<float> xPerNode;                     \textcolor{comment}{// vector for local portions}
  w.\hyperlink{classmpiworker_1_1MPIWorker_a24f713941043ab8d54574830a251995b}{scatterv}<\textcolor{keywordtype}{float}>(x,xPerNode,MPI::FLOAT);        \textcolor{comment}{// rank=0: \{ 1, 2, 3 \}}
                                                   \textcolor{comment}{// rank=1: \{ 4, 5, 6, 7 \}}
                                                   \textcolor{comment}{// rank=2: \{ 8, 9, 10, 11 \}}
  \textcolor{keywordflow}{for}( \textcolor{keyword}{auto} & e: xPerNode)
  \{
      e += w.\hyperlink{classmpiworker_1_1MPIWorker_acbd3bd07d15ffa90a2c112edeecee6e0}{getRankNode}();                        \textcolor{comment}{// rank=0: \{ 1, 2, 3 \}}
  \}                                                \textcolor{comment}{// rank=1: \{ 5, 6, 7, 8 \}}
                                                   \textcolor{comment}{// rank=2: \{ 10, 11, 12, 13 \}}
  std::vector<float> y( N );
  w.\hyperlink{classmpiworker_1_1MPIWorker_aff6b4d55cb55caa37c9a0cb08b3c5661}{gatherv}( xPerNode, y, MPI::FLOAT );            \textcolor{comment}{// rank=0: \{ 1, 2, 3, 5, 6, 7, 8, 10, 11, 12, 13
       \}}
                                                   \textcolor{comment}{// rank=1,2: \{ 0, 0, 0, 0, 0, 0, 0, 0, 0, 0, 0 \}}

  w.\hyperlink{classmpiworker_1_1MPIWorker_a49fda2aa379265e74f9b7504ad67ee9a}{allGatherv}( xPerNode, y, MPI::FLOAT );         \textcolor{comment}{// rank=0,1,2: \{ 1, 2, 3, 5, 6, 7, 8, 10, 11,
       12, 13 \}}
\end{DoxyCode}


Класс \hyperlink{classmpiworker_1_1MPIInit}{mpiworker\-::\-M\-P\-I\-Init} и функция calculate\-Portions играют вспомогательную роль, но могут быть использованы независимо от \hyperlink{classmpiworker_1_1MPIWorker}{mpiworker\-::\-M\-P\-I\-Worker}.

\subsubsection*{Класс \hyperlink{classmpiworker_1_1MPIInit}{mpiworker\-::\-M\-P\-I\-Init}}

Выполняет инициализацию и закрытие библиотеки M\-P\-I. Реализован как Singleton Meyers. Благодаря этому можно организовывать и хранить произвольное число схем разбиения массивов в программе\-: 
\begin{DoxyCode}
                                                \textcolor{comment}{// mpirun -np 3 ./example\_two\_mpiworkers}
std::vector<float> x1, x1PerNode,
                   x2, x2PerNode;
\textcolor{keywordtype}{int} N1, N2;

\hyperlink{classmpiworker_1_1MPIWorker}{mpiworker::MPIWorker} w1;                

\hyperlink{classmpiworker_1_1MPIWorker}{mpiworker::MPIWorker} w2;

\textcolor{keywordflow}{if}( !w1.\hyperlink{classmpiworker_1_1MPIWorker_acbd3bd07d15ffa90a2c112edeecee6e0}{getRankNode}() ) 
\{
    x1 = \{ 1, 2, 3, 4, 5 \}; 
    N1 = x1.size();
    x2 = \{ 6, 7, 8, 9, 10, 11, 12\};
    N2 = x2.size();
\}


w1.\hyperlink{classmpiworker_1_1MPIWorker_ae22bafa8bd4d6515e5b91ef95518c87d}{bcast}<\textcolor{keywordtype}{int}>(N1,MPI::INT);
w1.\hyperlink{classmpiworker_1_1MPIWorker_ae22bafa8bd4d6515e5b91ef95518c87d}{bcast}<\textcolor{keywordtype}{int}>(N2,MPI::INT);

w1.\hyperlink{classmpiworker_1_1MPIWorker_a2ff88f266efae23ec2a12a56bd0472d1}{setMode}(0);                                  \textcolor{comment}{// counts = \{ 0, 2, 3 \}}
w1.\hyperlink{classmpiworker_1_1MPIWorker_afcce321227c6d15a5fc2145cf59cd54d}{setNElems}(N1);                               \textcolor{comment}{// displs = \{ 0, 0, 2 \}}

w2.\hyperlink{classmpiworker_1_1MPIWorker_a2ff88f266efae23ec2a12a56bd0472d1}{setMode}(0);                                  \textcolor{comment}{// counts = \{ 0, 3, 4 \}}
w2.\hyperlink{classmpiworker_1_1MPIWorker_afcce321227c6d15a5fc2145cf59cd54d}{setNElems}(N2);                               \textcolor{comment}{// displs = \{ 0, 0, 3 \}}

w1.\hyperlink{classmpiworker_1_1MPIWorker_a24f713941043ab8d54574830a251995b}{scatterv}<\textcolor{keywordtype}{float}>(x1,x1PerNode,MPI::FLOAT);    \textcolor{comment}{// rank=0: \{\}}
                                                \textcolor{comment}{// rank=1: \{ 1, 2 \}}
                                                \textcolor{comment}{// rank=2: \{ 3, 4, 5 \}}
w2.\hyperlink{classmpiworker_1_1MPIWorker_a24f713941043ab8d54574830a251995b}{scatterv}<\textcolor{keywordtype}{float}>(x2,x2PerNode,MPI::FLOAT);    \textcolor{comment}{// rank=0: \{\}}
                                                \textcolor{comment}{// rank=1: \{ 6, 7, 8 \}}
                                                \textcolor{comment}{// rank=2: \{ 9, 10, 11, 12 \}}
\end{DoxyCode}


\subsubsection*{Функция \href{group__MPIWorker.html#ga6fd8303c1b4e39a4a623756fdcbeae6f}{\tt calculate\-Portions}}

Выполняет формирование вспомогательных массивов для деления некоторого общего количества элементов на приблизительно равные части коллективной операцией Scatterv или для объединения массивов значений на узлах в единый массив.

\subsubsection*{Доступные методы-\/обертки}

\begin{TabularC}{2}
\hline
\rowcolor{lightgray}\PBS\centering {\bf Метод M\-P\-I\-Worker }&\PBS\centering {\bf Операция коммуникатора M\-P\-I\-::\-C\-O\-M\-M\-\_\-\-W\-O\-R\-L\-D  }\\\cline{1-2}
\PBS\centering scatterv &\PBS\centering Scatterv \\\cline{1-2}
\PBS\centering all\-Gatherv &\PBS\centering Allgatherv \\\cline{1-2}
\PBS\centering gatherv &\PBS\centering Gatherv \\\cline{1-2}
\PBS\centering bcast &\PBS\centering Bcast \\\cline{1-2}
\PBS\centering reduce &\PBS\centering Reduce \\\cline{1-2}
\PBS\centering all\-Reduce &\PBS\centering Allreduce \\\cline{1-2}
\end{TabularC}
